% Options for packages loaded elsewhere
\PassOptionsToPackage{unicode}{hyperref}
\PassOptionsToPackage{hyphens}{url}
\PassOptionsToPackage{dvipsnames,svgnames,x11names}{xcolor}
%
\documentclass[
  letterpaper,
  DIV=11,
  numbers=noendperiod]{scrreprt}

\usepackage{amsmath,amssymb}
\usepackage{lmodern}
\usepackage{iftex}
\ifPDFTeX
  \usepackage[T1]{fontenc}
  \usepackage[utf8]{inputenc}
  \usepackage{textcomp} % provide euro and other symbols
\else % if luatex or xetex
  \usepackage{unicode-math}
  \defaultfontfeatures{Scale=MatchLowercase}
  \defaultfontfeatures[\rmfamily]{Ligatures=TeX,Scale=1}
\fi
% Use upquote if available, for straight quotes in verbatim environments
\IfFileExists{upquote.sty}{\usepackage{upquote}}{}
\IfFileExists{microtype.sty}{% use microtype if available
  \usepackage[]{microtype}
  \UseMicrotypeSet[protrusion]{basicmath} % disable protrusion for tt fonts
}{}
\makeatletter
\@ifundefined{KOMAClassName}{% if non-KOMA class
  \IfFileExists{parskip.sty}{%
    \usepackage{parskip}
  }{% else
    \setlength{\parindent}{0pt}
    \setlength{\parskip}{6pt plus 2pt minus 1pt}}
}{% if KOMA class
  \KOMAoptions{parskip=half}}
\makeatother
\usepackage{xcolor}
\setlength{\emergencystretch}{3em} % prevent overfull lines
\setcounter{secnumdepth}{5}
% Make \paragraph and \subparagraph free-standing
\ifx\paragraph\undefined\else
  \let\oldparagraph\paragraph
  \renewcommand{\paragraph}[1]{\oldparagraph{#1}\mbox{}}
\fi
\ifx\subparagraph\undefined\else
  \let\oldsubparagraph\subparagraph
  \renewcommand{\subparagraph}[1]{\oldsubparagraph{#1}\mbox{}}
\fi

\usepackage{color}
\usepackage{fancyvrb}
\newcommand{\VerbBar}{|}
\newcommand{\VERB}{\Verb[commandchars=\\\{\}]}
\DefineVerbatimEnvironment{Highlighting}{Verbatim}{commandchars=\\\{\}}
% Add ',fontsize=\small' for more characters per line
\usepackage{framed}
\definecolor{shadecolor}{RGB}{241,243,245}
\newenvironment{Shaded}{\begin{snugshade}}{\end{snugshade}}
\newcommand{\AlertTok}[1]{\textcolor[rgb]{0.68,0.00,0.00}{#1}}
\newcommand{\AnnotationTok}[1]{\textcolor[rgb]{0.37,0.37,0.37}{#1}}
\newcommand{\AttributeTok}[1]{\textcolor[rgb]{0.40,0.45,0.13}{#1}}
\newcommand{\BaseNTok}[1]{\textcolor[rgb]{0.68,0.00,0.00}{#1}}
\newcommand{\BuiltInTok}[1]{\textcolor[rgb]{0.00,0.23,0.31}{#1}}
\newcommand{\CharTok}[1]{\textcolor[rgb]{0.13,0.47,0.30}{#1}}
\newcommand{\CommentTok}[1]{\textcolor[rgb]{0.37,0.37,0.37}{#1}}
\newcommand{\CommentVarTok}[1]{\textcolor[rgb]{0.37,0.37,0.37}{\textit{#1}}}
\newcommand{\ConstantTok}[1]{\textcolor[rgb]{0.56,0.35,0.01}{#1}}
\newcommand{\ControlFlowTok}[1]{\textcolor[rgb]{0.00,0.23,0.31}{#1}}
\newcommand{\DataTypeTok}[1]{\textcolor[rgb]{0.68,0.00,0.00}{#1}}
\newcommand{\DecValTok}[1]{\textcolor[rgb]{0.68,0.00,0.00}{#1}}
\newcommand{\DocumentationTok}[1]{\textcolor[rgb]{0.37,0.37,0.37}{\textit{#1}}}
\newcommand{\ErrorTok}[1]{\textcolor[rgb]{0.68,0.00,0.00}{#1}}
\newcommand{\ExtensionTok}[1]{\textcolor[rgb]{0.00,0.23,0.31}{#1}}
\newcommand{\FloatTok}[1]{\textcolor[rgb]{0.68,0.00,0.00}{#1}}
\newcommand{\FunctionTok}[1]{\textcolor[rgb]{0.28,0.35,0.67}{#1}}
\newcommand{\ImportTok}[1]{\textcolor[rgb]{0.00,0.46,0.62}{#1}}
\newcommand{\InformationTok}[1]{\textcolor[rgb]{0.37,0.37,0.37}{#1}}
\newcommand{\KeywordTok}[1]{\textcolor[rgb]{0.00,0.23,0.31}{#1}}
\newcommand{\NormalTok}[1]{\textcolor[rgb]{0.00,0.23,0.31}{#1}}
\newcommand{\OperatorTok}[1]{\textcolor[rgb]{0.37,0.37,0.37}{#1}}
\newcommand{\OtherTok}[1]{\textcolor[rgb]{0.00,0.23,0.31}{#1}}
\newcommand{\PreprocessorTok}[1]{\textcolor[rgb]{0.68,0.00,0.00}{#1}}
\newcommand{\RegionMarkerTok}[1]{\textcolor[rgb]{0.00,0.23,0.31}{#1}}
\newcommand{\SpecialCharTok}[1]{\textcolor[rgb]{0.37,0.37,0.37}{#1}}
\newcommand{\SpecialStringTok}[1]{\textcolor[rgb]{0.13,0.47,0.30}{#1}}
\newcommand{\StringTok}[1]{\textcolor[rgb]{0.13,0.47,0.30}{#1}}
\newcommand{\VariableTok}[1]{\textcolor[rgb]{0.07,0.07,0.07}{#1}}
\newcommand{\VerbatimStringTok}[1]{\textcolor[rgb]{0.13,0.47,0.30}{#1}}
\newcommand{\WarningTok}[1]{\textcolor[rgb]{0.37,0.37,0.37}{\textit{#1}}}

\providecommand{\tightlist}{%
  \setlength{\itemsep}{0pt}\setlength{\parskip}{0pt}}\usepackage{longtable,booktabs,array}
\usepackage{calc} % for calculating minipage widths
% Correct order of tables after \paragraph or \subparagraph
\usepackage{etoolbox}
\makeatletter
\patchcmd\longtable{\par}{\if@noskipsec\mbox{}\fi\par}{}{}
\makeatother
% Allow footnotes in longtable head/foot
\IfFileExists{footnotehyper.sty}{\usepackage{footnotehyper}}{\usepackage{footnote}}
\makesavenoteenv{longtable}
\usepackage{graphicx}
\makeatletter
\def\maxwidth{\ifdim\Gin@nat@width>\linewidth\linewidth\else\Gin@nat@width\fi}
\def\maxheight{\ifdim\Gin@nat@height>\textheight\textheight\else\Gin@nat@height\fi}
\makeatother
% Scale images if necessary, so that they will not overflow the page
% margins by default, and it is still possible to overwrite the defaults
% using explicit options in \includegraphics[width, height, ...]{}
\setkeys{Gin}{width=\maxwidth,height=\maxheight,keepaspectratio}
% Set default figure placement to htbp
\makeatletter
\def\fps@figure{htbp}
\makeatother
\newlength{\cslhangindent}
\setlength{\cslhangindent}{1.5em}
\newlength{\csllabelwidth}
\setlength{\csllabelwidth}{3em}
\newlength{\cslentryspacingunit} % times entry-spacing
\setlength{\cslentryspacingunit}{\parskip}
\newenvironment{CSLReferences}[2] % #1 hanging-ident, #2 entry spacing
 {% don't indent paragraphs
  \setlength{\parindent}{0pt}
  % turn on hanging indent if param 1 is 1
  \ifodd #1
  \let\oldpar\par
  \def\par{\hangindent=\cslhangindent\oldpar}
  \fi
  % set entry spacing
  \setlength{\parskip}{#2\cslentryspacingunit}
 }%
 {}
\usepackage{calc}
\newcommand{\CSLBlock}[1]{#1\hfill\break}
\newcommand{\CSLLeftMargin}[1]{\parbox[t]{\csllabelwidth}{#1}}
\newcommand{\CSLRightInline}[1]{\parbox[t]{\linewidth - \csllabelwidth}{#1}\break}
\newcommand{\CSLIndent}[1]{\hspace{\cslhangindent}#1}

\KOMAoption{captions}{tableheading}
\makeatletter
\@ifpackageloaded{tcolorbox}{}{\usepackage[many]{tcolorbox}}
\@ifpackageloaded{fontawesome5}{}{\usepackage{fontawesome5}}
\definecolor{quarto-callout-color}{HTML}{909090}
\definecolor{quarto-callout-note-color}{HTML}{0758E5}
\definecolor{quarto-callout-important-color}{HTML}{CC1914}
\definecolor{quarto-callout-warning-color}{HTML}{EB9113}
\definecolor{quarto-callout-tip-color}{HTML}{00A047}
\definecolor{quarto-callout-caution-color}{HTML}{FC5300}
\definecolor{quarto-callout-color-frame}{HTML}{acacac}
\definecolor{quarto-callout-note-color-frame}{HTML}{4582ec}
\definecolor{quarto-callout-important-color-frame}{HTML}{d9534f}
\definecolor{quarto-callout-warning-color-frame}{HTML}{f0ad4e}
\definecolor{quarto-callout-tip-color-frame}{HTML}{02b875}
\definecolor{quarto-callout-caution-color-frame}{HTML}{fd7e14}
\makeatother
\makeatletter
\makeatother
\makeatletter
\@ifpackageloaded{bookmark}{}{\usepackage{bookmark}}
\makeatother
\makeatletter
\@ifpackageloaded{caption}{}{\usepackage{caption}}
\AtBeginDocument{%
\ifdefined\contentsname
  \renewcommand*\contentsname{Table of contents}
\else
  \newcommand\contentsname{Table of contents}
\fi
\ifdefined\listfigurename
  \renewcommand*\listfigurename{List of Figures}
\else
  \newcommand\listfigurename{List of Figures}
\fi
\ifdefined\listtablename
  \renewcommand*\listtablename{List of Tables}
\else
  \newcommand\listtablename{List of Tables}
\fi
\ifdefined\figurename
  \renewcommand*\figurename{Figure}
\else
  \newcommand\figurename{Figure}
\fi
\ifdefined\tablename
  \renewcommand*\tablename{Table}
\else
  \newcommand\tablename{Table}
\fi
}
\@ifpackageloaded{float}{}{\usepackage{float}}
\floatstyle{ruled}
\@ifundefined{c@chapter}{\newfloat{codelisting}{h}{lop}}{\newfloat{codelisting}{h}{lop}[chapter]}
\floatname{codelisting}{Listing}
\newcommand*\listoflistings{\listof{codelisting}{List of Listings}}
\makeatother
\makeatletter
\@ifpackageloaded{caption}{}{\usepackage{caption}}
\@ifpackageloaded{subcaption}{}{\usepackage{subcaption}}
\makeatother
\makeatletter
\@ifpackageloaded{tcolorbox}{}{\usepackage[many]{tcolorbox}}
\makeatother
\makeatletter
\@ifundefined{shadecolor}{\definecolor{shadecolor}{rgb}{.97, .97, .97}}
\makeatother
\makeatletter
\makeatother
\ifLuaTeX
  \usepackage{selnolig}  % disable illegal ligatures
\fi
\IfFileExists{bookmark.sty}{\usepackage{bookmark}}{\usepackage{hyperref}}
\IfFileExists{xurl.sty}{\usepackage{xurl}}{} % add URL line breaks if available
\urlstyle{same} % disable monospaced font for URLs
\hypersetup{
  pdftitle={Kom i gang med DAPLA},
  pdfauthor={Øyvind Bruer-Skarsbø},
  colorlinks=true,
  linkcolor={blue},
  filecolor={Maroon},
  citecolor={Blue},
  urlcolor={Blue},
  pdfcreator={LaTeX via pandoc}}

\title{Kom i gang med DAPLA}
\author{Øyvind Bruer-Skarsbø}
\date{10/9/2022}

\begin{document}
\maketitle
\ifdefined\Shaded\renewenvironment{Shaded}{\begin{tcolorbox}[interior hidden, boxrule=0pt, frame hidden, enhanced, borderline west={3pt}{0pt}{shadecolor}, sharp corners, breakable]}{\end{tcolorbox}}\fi

\renewcommand*\contentsname{Innhold}
{
\hypersetup{linkcolor=}
\setcounter{tocdepth}{2}
\tableofcontents
}
\bookmarksetup{startatroot}

\hypertarget{velkommen}{%
\chapter*{Velkommen}\label{velkommen}}
\addcontentsline{toc}{chapter}{Velkommen}

DAPLA står for dataplattform og er SSBs nye plattform for
statistikkproduksjon. Arbeidet startet som et utviklingsprosjekt i 2018
i sammenheng med Skatteetatens prosjekt \emph{Sirius}. Idag er
plattformen mer moden og klar for å ta imot flere statistikker. Denne
boken er ment som

DAPLA står for dataplattform og er SSBs nye plattform for
statistikkproduksjon. Arbeidet startet som et utviklingsprosjekt i 2018
i sammenheng med Skatteetatens prosjekt \emph{Sirius}. Idag er
plattformen mer moden og klar for å ta imot flere statistikker. Denne
boken er ment som

\begin{tcolorbox}[enhanced jigsaw, colback=white, rightrule=.15mm, opacityback=0, bottomrule=.15mm, leftrule=.75mm, arc=.35mm, toprule=.15mm, colframe=quarto-callout-note-color-frame, left=2mm, breakable]
\begin{minipage}[t]{5.5mm}
\textcolor{quarto-callout-note-color}{\faInfo}
\end{minipage}%
\begin{minipage}[t]{\textwidth - 5.5mm}
Denne boken er skrevet med \href{https://quarto.org/}{Quarto} og er
publisert på \url{https://statisticsnorway.github.io/dapla-manual/}.
Alle ansatte i SSB kan bidra til boken ved klone
\href{https://github.com/statisticsnorway/dapla-manual}{dette repoet},
gjøre endringer i en branch, og sende en pull request til
administratorene av repoet (Team Statistikktjenester).\end{minipage}%
\end{tcolorbox}

\bookmarksetup{startatroot}

\hypertarget{forord}{%
\chapter*{Forord}\label{forord}}
\addcontentsline{toc}{chapter}{Forord}

Denne boken vil la SSB-ansatte ta i bruk grunnleggende funksjonalitet på
DAPLA uten hjelp fra andre.

\part{Introduksjon}

Målet med dette kapitlet er å gi en grunnleggende innføring i hva som
legges i ordet \textbf{Dapla}. I tillegg gis en forklaring på hvorfor
disse valgene er tatt.

\hypertarget{hva-er-dapla}{%
\chapter{Hva er Dapla?}\label{hva-er-dapla}}

\hypertarget{hvorfor-dapla}{%
\chapter{Hvorfor Dapla?}\label{hvorfor-dapla}}

\hypertarget{arkitektur}{%
\chapter{Arkitektur}\label{arkitektur}}

Hvilke komponenter er plattformen bygd opp på? Forklart på lettest mulig
måte.

\hypertarget{innlogging}{%
\chapter{Innlogging}\label{innlogging}}

\hypertarget{jupyterlab}{%
\chapter{Jupyterlab}\label{jupyterlab}}

\hypertarget{bakke-vs.-sky}{%
\chapter{Bakke vs.~sky}\label{bakke-vs.-sky}}

\part{Opprette Dapla-team}

\hypertarget{hva-er-dapla-team}{%
\chapter{Hva er Dapla-team?}\label{hva-er-dapla-team}}

Mer kommer

\hypertarget{opprette-dapla-team}{%
\chapter{Opprette Dapla-team}\label{opprette-dapla-team}}

\hypertarget{google-cloud-console}{%
\chapter{Google Cloud Console}\label{google-cloud-console}}

\hypertarget{lagre-data}{%
\chapter{Lagre data}\label{lagre-data}}

\hypertarget{hente-data}{%
\chapter{Hente data}\label{hente-data}}

\hypertarget{fra-bakke-til-sky}{%
\chapter{Fra bakke til sky}\label{fra-bakke-til-sky}}

\hypertarget{administrasjon-av-team}{%
\chapter{Administrasjon av team}\label{administrasjon-av-team}}

\part{Beste-praksis for koding}

\hypertarget{ssb-project}{%
\chapter{SSB-project}\label{ssb-project}}

Fremtidens produksjonsløp på \textbf{Dapla} bør følge noen helt klare
retningslinjer for arbeidsprosesser og kode. Dette bør blant annet
inkludere:

\begin{enumerate}
\def\labelenumi{\arabic{enumi}.}
\tightlist
\item
  \textbf{Standard mappestruktur}\\
  En standard mappestruktur gjør det lettere å dele og samarbeide om
  kode, som igjen reduserer sårbarheten knyttet til at få personer
  kjenner koden.
\item
  \textbf{Virtuelt miljø}\\
  Virtuelle miljøer isoloerer og lagrer informasjon knyttet til kode.
  For at publiserte tall skal være reproduserbare er SSB avhengig av at
  blant annet pakkeversjoner og versjon av Python/R lagres sammen med
  kode som er kjørt.
\item
  \textbf{Versjonshåndtering med Git}\\
  Versjonshåndtering av kode er svært viktig for å kunne gjenskape og
  samarbeide om kode. \href{https://git-scm.com/}{Git} er
  verdensstandarden for å gjøre dette, og derfor legges det opp til at
  all kode skal versjonshåndteres med Git i SSB.
\item
  \textbf{Lagre kode på Github}\\
  På Dapla er det ingen fellesmappe som alle i SSB har tilgang til og
  hvor vi kan dele kode slik vi har gjort i bakkemiljøet tidligere. Kode
  som er versjonshåndtert med Git bruker som regel et remote
  repo\footnote{\emph{Remote repo} er en felle mappe som er lagret på en
    annen maskin.
    \href{https://git-scm.com/book/en/v2/Git-Basics-Working-with-Remotes}{Les
    mer her.}} som er spesialsydd for Git og som skal deles med resten
  av verden hvis man ønsker. I SSB har vi valgt å bruke GitHub, der SSB
  har et eget område som heter
  \href{https://github.com/statisticsnorway}{statisticsnorway}.
\end{enumerate}

\href{https://statistics-norway.atlassian.net/wiki/spaces/STAT/overview?homepageId=3127312686}{Team
Statistikktjenester} har laget en CLI\footnote{CLI =
  Command-Line-Interface. Dvs. et program som er skrevet for å brukes
  terminalen ved hjelp av enkle kommandoer.} som skal gjøre dette lett å
implemententere dette i kode. Den heter
\href{https://github.com/statisticsnorway/ssb-project-cli}{ssb-project}
og hjelper deg implementere det som til enhver tid er beste-praksis for
koding.

Under vises det hvordan man bruker \texttt{ssb-project} til sette opp et
prosjekt. Men programmet forutsettet at du har en GitHub-bruker som er
knyttet opp mot
\href{https://github.com/statisticsnorway}{statisticsnorway}. De første
underkapitlene er derfor en beskrivelse av dette.

\hypertarget{opprett-github-bruker}{%
\section{Opprett GitHub-bruker}\label{opprett-github-bruker}}

Dette kapitlet er bare relevant hvis man ikke har en GitHub-brukerkonto
fra før. For å bruke \texttt{ssb-project}-programmet til å generere et
\textbf{remote repo} på GitHub må du ha en konto. Derfor starter vi med
å gjøre dette. Det er en engangsjobb og du trenger aldri gjøre det
igjen.

\begin{tcolorbox}[enhanced jigsaw, colback=white, rightrule=.15mm, opacityback=0, bottomrule=.15mm, leftrule=.75mm, arc=.35mm, toprule=.15mm, colframe=quarto-callout-note-color-frame, left=2mm, breakable]
\begin{minipage}[t]{5.5mm}
\textcolor{quarto-callout-note-color}{\faInfo}
\end{minipage}%
\begin{minipage}[t]{\textwidth - 5.5mm}
SSB har valgt å ikke sette opp SSB-brukerne til de ansatte som
GitHub-brukere. En viktig årsak er at er en GitHub-konto ofte regnes som
en del av den ansattes CV. For de som aldri har brukt GitHub før kan det
virke fremmed, men det er nok en fordel på sikt når alle blir godt kjent
med denne arbeidsformen.\end{minipage}%
\end{tcolorbox}

Slik gjør du det:

\begin{enumerate}
\def\labelenumi{\arabic{enumi}.}
\tightlist
\item
  Gå til \url{https://github.com/}
\item
  Trykk \textbf{Sign up} øverst i høyre hjørne
\item
  Svar på spørsmålene du blir stilt.
\end{enumerate}

Husk at du lager en personlig konto uavhengig av SSB. Brukernavnet kan
være noe annet enn brukernavnet ditt i SSB. I neste steg skal vi knytte
denne kontoen til din SSB-bruker.

\hypertarget{koble-seg-til-ssb}{%
\section{Koble seg til SSB}\label{koble-seg-til-ssb}}

Hvis du har fullført forrige steg så har du nå en SSB-konto. Hvis du
står på din profil-side så ser den slik ut:

\hypertarget{autentifisering}{%
\section{Autentifisering}\label{autentifisering}}

\hypertarget{ssb-project-cli}{%
\section{ssb-project-cli}\label{ssb-project-cli}}

\hypertarget{git-og-github}{%
\chapter{Git og Github}\label{git-og-github}}

\hypertarget{virtuelle-miljuxf8er}{%
\chapter{Virtuelle miljøer}\label{virtuelle-miljuxf8er}}

\hypertarget{python}{%
\section{Python}\label{python}}

Poetry bør benyttes for python. Mer kommer

\hypertarget{r}{%
\section{R}\label{r}}

\hypertarget{jupyter-kernels}{%
\chapter{Jupyter-kernels}\label{jupyter-kernels}}

\hypertarget{installere-pakker}{%
\chapter{Installere pakker}\label{installere-pakker}}

\hypertarget{python-1}{%
\section{Python}\label{python-1}}

Mer kommer.

\hypertarget{installering}{%
\subsection{Installering}\label{installering}}

\hypertarget{avinstallering}{%
\subsection{Avinstallering}\label{avinstallering}}

\hypertarget{oppgradere-pakker}{%
\subsection{Oppgradere pakker}\label{oppgradere-pakker}}

\hypertarget{r-1}{%
\section{R}\label{r-1}}

\hypertarget{installering-1}{%
\subsection{Installering}\label{installering-1}}

\hypertarget{avinstallering-1}{%
\subsection{Avinstallering}\label{avinstallering-1}}

\hypertarget{oppgradere-pakker-1}{%
\subsection{Oppgradere pakker}\label{oppgradere-pakker-1}}

\hypertarget{samarbeid}{%
\chapter{Samarbeid}\label{samarbeid}}

Noen har opprettet et ssb-project og pushet til Github. Hvordan skal
kollegaer gå frem for å bidra inn i koden?

\hypertarget{vedlikehold}{%
\chapter{Vedlikehold}\label{vedlikehold}}

\part{Jupyterlab på bakken}

\hypertarget{installere-pakker-1}{%
\chapter{Installere pakker}\label{installere-pakker-1}}

\hypertarget{python-2}{%
\section{Python}\label{python-2}}

Installering av pakker i Jupyter miljøer på bakken (f.eks
\url{https://sl-jupyter-p.ssb.no}) foregår stort sett helt lik
\href{./pakke-install.html}{som på Dapla}. Det er én viktig forskjell,
og det er at installasjon skjer via en proxy som heter Nexus.

\hypertarget{pip}{%
\subsection{Pip}\label{pip}}

Pip er ferdig konfigurert for bruk av Nexus og kan kjøres som
\href{./pakke-install.html}{beskrevet for Dapla}

\hypertarget{poetry}{%
\subsection{Poetry}\label{poetry}}

Hvis man bruker Poetry for håndtering av pakker i et prosjekt, så må man
kjøre følgende kommando i prosjekt-mappe etter prosjektet er opprettet.

\begin{Shaded}
\begin{Highlighting}[]
\NormalTok{poetry source add {-}{-}default nexus \textasciigrave{}echo $PIP\_INDEX\_URL\textasciigrave{}}
\end{Highlighting}
\end{Shaded}

Da får man installere pakker som vanlig f.eks

\begin{Shaded}
\begin{Highlighting}[]
\NormalTok{poetry add matplotlib}
\end{Highlighting}
\end{Shaded}

\begin{tcolorbox}[enhanced jigsaw, colback=white, rightrule=.15mm, opacityback=0, bottomrule=.15mm, leftrule=.75mm, arc=.35mm, toprule=.15mm, colframe=quarto-callout-warning-color-frame, left=2mm, breakable]
\begin{minipage}[t]{5.5mm}
\textcolor{quarto-callout-warning-color}{\faExclamationTriangle}
\end{minipage}%
\begin{minipage}[t]{\textwidth - 5.5mm}

Hvis man forsøker å installere prosjektet i et annet miljø (f.eks
Dapla), så må man fjerner \texttt{nexus} kilden ved å kjøre

\begin{Shaded}
\begin{Highlighting}[]
\NormalTok{poetry source remove nexus}
\end{Highlighting}
\end{Shaded}

\end{minipage}%
\end{tcolorbox}

\hypertarget{r-2}{%
\section{R}\label{r-2}}

\hypertarget{installering-2}{%
\subsection{Installering}\label{installering-2}}

\hypertarget{avinstallering-2}{%
\subsection{Avinstallering}\label{avinstallering-2}}

\hypertarget{oppgradere-pakker-2}{%
\subsection{Oppgradere pakker}\label{oppgradere-pakker-2}}

\hypertarget{lese-inn-filer}{%
\chapter{Lese inn filer}\label{lese-inn-filer}}

Mer kommer.

\hypertarget{sas7bdat}{%
\section{sas7bdat}\label{sas7bdat}}

\hypertarget{oracle}{%
\section{Oracle}\label{oracle}}

\hypertarget{fame}{%
\section{Fame}\label{fame}}

\hypertarget{tekstfiler}{%
\section{Tekstfiler}\label{tekstfiler}}

\hypertarget{parquet}{%
\section{Parquet}\label{parquet}}

\part{Avansert}

\hypertarget{ideer}{%
\chapter{IDE'er}\label{ideer}}

Forklare situasjonen nå. Kun Jupyterlab. Kan kjøre remote session med
Rstudio, Pycharm og VSCode.

\hypertarget{rstudio}{%
\section{RStudio}\label{rstudio}}

\hypertarget{vscode}{%
\section{VSCode}\label{vscode}}

\hypertarget{pycharm}{%
\section{Pycharm}\label{pycharm}}

\hypertarget{schedulering}{%
\chapter{Schedulering}\label{schedulering}}

\hypertarget{databaser}{%
\chapter{Databaser}\label{databaser}}

\hypertarget{bigquery}{%
\section{BigQuery}\label{bigquery}}

\hypertarget{cloudsql}{%
\section{CloudSQL}\label{cloudsql}}

\bookmarksetup{startatroot}

\hypertarget{referanser}{%
\chapter*{Referanser}\label{referanser}}
\addcontentsline{toc}{chapter}{Referanser}

\hypertarget{refs}{}
\begin{CSLReferences}{0}{0}
\end{CSLReferences}



\end{document}
